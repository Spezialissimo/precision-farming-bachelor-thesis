\documentclass[12pt,a4paper,openright,twoside]{book}
\usepackage[utf8]{inputenc}
\usepackage{disi-thesis}
\usepackage{code-lstlistings}
\usepackage{notes}
\usepackage{shortcuts}
\usepackage{acronym}

\school{\unibo}
\programme{Corso di Laurea Triennale in Ingegneria e Scienze Informatiche}
\title{Fancy Title}
\author{Candidate Name}
\date{\today}
\subject{Supervisor's course name}
\supervisor{Prof. Supervisor Here}
\cosupervisor{Dott. CoSupervisor 1}
\morecosupervisor{Dott. CoSupervisor 2}
\session{I}
\academicyear{2023-2024}

% Definition of acronyms
\acrodef{IoT}{Internet of Thing}

\mainlinespacing{1.241} % line spacing in mainmatter, comment to default (1)

\begin{document}

\frontmatter\frontispiece

\begin{abstract}
    TODO
\end{abstract}

\begin{dedication} % this is optional
    Optional. Max a few lines.
\end{dedication}

%----------------------------------------------------------------------------------------
\tableofcontents
%\listoffigures     % (optional) comment if empty
%\lstlistoflistings % (optional) comment if empty

\mainmatter

\chapter*{Introduzione}
\addcontentsline{toc}{chapter}{Introduzione}  % Aggiunge l'Introduzione al sommario
\setcounter{chapter}{0}  % Resetta il contatore dei capitoli a 0


Generalmente viene scritta alla fine, non è un capitolo, racchiude un riassunto stretto del contenuto della tesi: problema -> soluzioni esistenti -> descrizione nuova soluzione -> conclusioni. Lunghezza da 1-3 .

Write your intro here.
You can use acronyms that your defined previously,
such as \ac{IoT}.
%
If you use acronyms twice,
they will be written in full only once
(indeed, you can mention the \ac{IoT} now without it being fully explained).

\paragraph{Structure of the Thesis}

\note{At the end, describe the structure of the paper}

\chapter{Precision Watering}

Introduce il lettore alla tesi, descrive il concetto di agricoltura di precisione, perché è nata (in termini di necessità), quali obbiettivi si pone e quali sono i maggiori limiti che sta incontrando; in allegato trovi la mia tesi magistrale, sentiti pure libero di usarlo come fonte, soprattutto il primo capitolo. Lunghezza 10-15 pagine


\chapter{Tecnologie utilizzate}

%Descrive appunto le tecnologie utilizzate nel progetto: sia lato HW (Arduino + Raspberry + sensori), sia lato SW (stack tecnologico). Lunghezza 5-8 pagine.


\section{Hardware}

\begin{itemize}
    \item \textbf{Sensori}, TODO numero sensore e citazione alle specifiche
    \item \textbf{Arduino}, TODO
    \item \textbf{Raspberry}, TODO
\end{itemize}
    
\section{Software}

    \begin{itemize}
        \item \textbf{JavaScript}, TODO
        \begin{itemize}
            \item \textbf{Chart.js}, TODO
            \begin{itemize}
                \item \textbf{chartjs-chart-matrix}, TODO
                \item \textbf{chartjs-plugin-streaming}, TODO
                \item \textbf{chartjs-plugin-datalabels}, TODO
                \item \textbf{chartjs-adapter-luxon}, TODO
            \end{itemize}        
            \item \textbf{luxon}, TODO
            \item \textbf{bootstrap}, TODO
            \item \textbf{jQuery}, TODO
            \item \textbf{Socket.io}, TODO
        \end{itemize}
        \item \textbf{Python}, TODO
        \begin{itemize}
            \item \textbf{Flask}, TODO
            \item \textbf{Flask-SocketIO}, TODO
            \item \textbf{numpy}, TODO
            \item \textbf{pyserial}, TODO
            \item \textbf{dotenv}, TODO
            \item \textbf{pytz}, TODO
            \item \textbf{scipy}, TODO
        \end{itemize}
        \item \textbf{C++}, TODO
        \begin{itemize}
            \item \textbf{Wiring}, TODO
            \item \textbf{TimerInterrupt}, TODO
            \item \textbf{ArduinoJson}, TODO
        \end{itemize}        
    \end{itemize}


\chapter{Un prototipo di irrigazione prescrittivo}
Descrive nel dettaglio il tuo lavoro, quindi come le tecnologie presentate nella precedente sezione sono state utilizzate nel progetto. Puoi prenderla larga descrivendo il nostro sistema (quello che ti ho mostrato sul mio pc) per poi descrivere la necessità di averne una rappresentazione su piccola scala e come è stata realizzata, così come descrivere il ciclo di vita del dato (collection, processing, exploitation). Ora come ora è normale tu non abbia chiare le idee su questo capitolo, andando avanti integreremo le tue conoscenze con quelle pregresse nostre sul dominio applicativo e sul nostro sistema. Questo capitolo deve essere il core della tesi, lunghezza 20 pagine.

\chapter{Conclusioni e sviluppi futuri}

Breve capitolo che trae le conclusioni sul lavoro svolto, il risultato ottenuto rispetto a quello atteso e lo spazio dedicato a migliorie future.

%
% BIBLIOGRAPHY
%

\backmatter

\nocite{*} % Remove this as soon as you have the first citation

\bibliographystyle{alpha}
\bibliography{bibliography}

\begin{acknowledgements} % this is optional
    Optional. Max 1 page.
\end{acknowledgements}

\end{document}
