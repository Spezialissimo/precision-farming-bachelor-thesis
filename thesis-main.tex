\documentclass[12pt,a4paper,openright,twoside]{book}
\usepackage[utf8]{inputenc}
\usepackage{disi-thesis}
\usepackage{code-lstlistings}
\usepackage{notes}
\usepackage{shortcuts}
\usepackage{acronym}

\school{\unibo}
\programme{Corso di Laurea Triennale in Ingegneria e Scienze Informatiche}
\title{Prototipo d'irrigazione prescrittiva}
\author{Davide Speziali}
\date{\today}
\subject{Supervisor's course name}
\supervisor{Prof. Supervisor Here}
\cosupervisor{Dott. CoSupervisor 1}
\morecosupervisor{Dott. CoSupervisor 2}
\session{I}
\academicyear{2023-2024}

% Definition of acronyms
\acrodef{IoT}{Internet of Thing}
\acrodef{GPIO}{General Purpose Input/Output}
\acrodef{PWM}{pulse-width modulation}
\acrodef{SBC}{Single Board Computer}
\acrodef{JS}{JavaScript}
\acrodef{UN}{United Nations}
\acrodef{SDG}{Sustainable Development Goals}
\acrodef{FC}{Field Capacity}
\acrodef{PWP}{Permanent Wilting Point}
\acrodef{SWRC}{Soil Water Retention Curve}
\acrodef{SWC}{Soil Water Content}
\acrodef{SWP}{Soil Water Potential}
\acrodef{$ET_0$}{potential evapotranspiration}
\acrodef{$ET_a$}{actual evapotranspiration}
\acrodef{$K_c$}{crop coefficient}
\acrodef{GPS}{Global Positioning System}

\mainlinespacing{1.241} % line spacing in mainmatter, comment to default (1)

\begin{document}

\frontmatter\frontispiece

\begin{abstract}
    TODO
\end{abstract}

\begin{dedication} % this is optional
    Optional. Max a few lines.
\end{dedication}

%----------------------------------------------------------------------------------------
\tableofcontents
%\listoffigures     % (optional) comment if empty
%\lstlistoflistings % (optional) comment if empty

\mainmatter

\chapter*{Introduzione}
\addcontentsline{toc}{chapter}{Introduzione}
\setcounter{chapter}{0}

%Generalmente viene scritta alla fine, non è un capitolo, racchiude un riassunto stretto del contenuto della tesi: problema -> soluzioni esistenti -> descrizione nuova soluzione -> conclusioni. Lunghezza da 1-3 .
%Write your intro here.
%You can use acronyms that your defined previously,
%such as \ac{IoT}.
%
%If you use acronyms twice,
%they will be written in full only once
%(indeed, you can mention the \ac{IoT} now without it being fully explained).

\paragraph{Structure of the Thesis}

\note{At the end, describe the structure of the paper}

\chapter{Precision Watering}

La risposta al cambiamento climatico sta diventando uno dei temi più importanti per il futuro delle popolazioni del pianeta.
Nel 2015 le \ac{UN} hanno adottato "l'agenda per lo sviluppo sostenibile 2030", con l'obiettivo di rispondere in maniera sistematica al problema del cambiamento climatico.
Vengono specificate 16 \ac{SDG}: obbiettivi specifici che ogni nazione deve raggiungere. Tra essi \ac{SDG} 6 "Acqua pulita e sanificazione", si concentra sul garantire un accesso sicuro e sostenibile alle risorse idriche \cite{SDG-6}. Il punto 6.4 in particolare punta a un miglioramento dell'efficienza dell'uso di acqua in tutti i settori \cite{SDG-6-4-1,UN-WATER-WATER-USE-EFF-21}.
%
Il settore agricolo è il più importante per rendere possibile questa transizione: la quantità di acqua usata per l'agricoltura quasi raddoppiata dal 1970 e oggi l'agricoltura rimane l'area dove l'utilizzo di acqua è il maggiore, circa tre volte rispetto quello delle altre industrie. \cite{FAO-AQUASTAT-2020}
%
È dunque aumentato l'interesse per lo sviluppo di tecnologie di monitoraggio dei terreni coltivati e lo sviluppo di modelli per l'irrigazione efficiente e automatica.

La \cref{introduzione-agricoltura-precisione} introduce il concetto di agricoltura di precisione come strumento per l'ottimizzazione delle risorse, la \cref{componenti-chiave} elenca le sue principali componenti, la \cref{esempi-tecnologie} illustra alcuni esempi di tecnologie utilizzate, mentre la \cref{limiti} copre i limiti attuali di queste tecnologie.

\section{Introduzione generale all’agricoltura di precisione}\label{introduzione-agricoltura-precisione}

\subsection{Agricoltura e cambiamento climatico}

L'agricoltura è una delle attività più importanti per l'essere umano, ha subito negli anni enormi trasformazioni e l'era moderna non fa differenza\cite{FEDERICO-FEEDING-2005}. Oggi è infatti interessata da continui progressi tecnologici e cambiamenti sociali, economici e ambientali. Anche grazie a innovazioni tecnologiche, la produzione è in stabile aumento, permettendo a tutte le popolazioni del mondo di accedere a una maggiore varietà di prodotti \cite{FAO-AGRO-STATS-2023}.

La ricerca di metodi più sostenibili per la gestione del settore agricolo è diventato un obbiettivo primario per affrontare nuove sfide a livello mondiale \cite{SDG-2-4-1}. In primis il costante aumento della popolazione, che si stima raggiungerà i 10 miliardi persone verso gli anni 2080\cite{UN-POPULATION-2024}, porterà un aumento della richiesta di prodotti agricoli e quindi la necessità di un aumento dell'offerta. Inoltre, diventa sempre più necessario agire in tema d'impatto che il settore ha sull'ambiente, come deforestazione, uso eccessivo di risorse idriche e fertilizzanti inquinanti \cite{EIOM-2012}. Questi sono tutti elementi che hanno effetto sull'aumento delle temperature, che a loro volta aumentano la probabilità di eventi meteorologici estremi che possono rovinare o totalmente distruggere i raccolti: in Cina cambiamento climatico potrebbe far diminuire le rese di riso, grano e mais rispettivamente del 36,25\%, 18,26\% e 45,10\% entro la fine di questo secolo\cite{ZHANG-PENG-2017}; in Italia sono studiati gli effetti di eventi di siccità sul territorio.

L'agricoltura, basandosi su conoscenze maturate nel corso di millenni, è particolarmente fragile al rapido cambiamento delle variabili climatiche: diventa quindi necessario riuscire a parametrizzare l'ambiente di coltivazione, capire quali parametri sono realmente importanti per le culture e studiare come possono essere monitorate, in modo che gli agricoltori abbiano gli strumenti per fronteggiare questa fragilità.

\subsection{Parametri di coltivazione}

Vengono definite proprietà idrauliche del suolo una serie di parametri che permettono di definire il rapporto tra suolo e acqua. Questi parametri possono essere tantissimi e dipendere dal contesto che si vuole analizzare.
In generale, possiamo vedere il suolo come la riserva idrica della pianta, l'obbiettivo dell'irrigazione deve essere, quindi, quello di somministrare la giusta quantità di acqua, evitando quindi di avere un terreno troppo bagnato o troppo asciutto.

La quantità di acqua presente all'interno del terreno viene rappresentata dal \ac{SWC}, tipicamente misurato in percentuale di volume di acqua rispetto al volume del suolo.
In FIGURA possiamo vedere una serie di valori "soglia" che \ac{SWC} può raggiungere.
Il termine \textit{saturation} indica uno stato in cui all'interno del terreno non è più presente l'aria, e ogni poro è quindi pieno di acqua. Tipicamente, fatta eccezione per certi tipi di culture, come quella del riso, questo stato non è adatto alla crescita della pianta.
Lo stato \textit{oven dry}, al contrario, avviene in totale assenza di acqua, non è quindi adatto alla crescita di nessun tipo di pianta.
Viene definito \ac{FC} il valore di \ac{SWC} massimo che permette alla pianta di assorbire l'acqua dal terreno. In questo stato i pori più grandi hanno al loro interno sia aria che acqua, mentre quelli più piccoli sono pieni di acqua.
Infine, \ac{PWP} è invece il punto dopo il quale la pianta non riesce più a estrarre acqua dal terreno, per quanto sia ancora in minima parte presente.
Questi valori ci permettono di definire quale potrebbe essere l'irrigazione adatta, si vuole infatti che \ac{SWC} sia compreso tra \ac{FC} e \ac{PWP}.

Un ulteriore parametro idraulico del suolo è \ac{SWP}, potenziale idrico del suolo, rappresenta l'energia necessaria per estrarre l'acqua dal suolo e indica quindi la disponibilità di acqua delle piante. Viene espresso in termini di pressione e comprende diverse componenti che contribuiscono al valore finale. Le principali sono:

\begin{itemize}
    \item \textbf{Potenziale gravitazionale}: deriva dalla forza di gravità e rappresenta l'energia potenziale dovuta alla sua posizione verticale.
    \item \textbf{Potenziale matriciale}:  rappresenta l'energia necessaria per estrarre l'acqua dal suolo, risultante dalle forze d'interazione tra l'acqua e le particelle del suolo. È dovuto a forze come adesione e capillarità.
    \item \textbf{Potenziale osmotico}: rappresenta l'energia associata alla concentrazione di soluti e alla tendenza dell'acqua a muoversi da aree con alto potenziale osmotico (bassa concentrazione di soluti) ad aree con basso potenziale osmotico (alta concentrazione di soluti), come le radici delle piante.
\end{itemize}

La relazione tra \ac{SWP} e \ac{SWC} è definita dal concetto di \ac{SWRC}, curva di ritenzione idrica del suolo: la quantità di acqua trattenuta nel suolo a diversi livelli di pressione.

Tutte questi parametri dipendono da una serie di fattori, come:
\begin{itemize}
    \item \textbf{Proprietà fisiche del terreno}, caratteristiche come la tessitura (distribuzione delle dimensioni delle particelle del suolo) e la struttura (raggruppamento in composti porosi delle particelle del suolo) hanno peso nella definizione di \ac{SWP}, e quindi chiaramente anche di \ac{SWRC}. In FIGURA si possono vedere delle classificazioni comuni per i tipi di terreni in base alla tessitura, mentre in FIGURA in base alla struttura. In FIGURA sono presenti esempi di \ac{SWRC} per diversi tipi di terreno.
    \item \textbf{Cultura}, ovviamente ogni tipo di pianta ha una propria \ac{FC} e un proprio \ac{PWP}. Inoltre l'apparato radicale di una pianta ha a sua volta effetto sul \ac{SWP} e quindi, come definito precedentemente, sulla \ac{SWRC}.
\end{itemize}

È inoltre importante notare che sia l'apparato radicale che le proprietà fisiche del suolo non sono costanti all'interno dello stesso campo. A causa della quantità di variabili che possono modificare sostanzialmente la \ac{SWRC}, è molto difficile sapere di per certo la quantità d'irrigazione richiesta all'interno dell'intero campo senza monitorare attivamente l'andamento dell'acqua al suo interno.

Un ulteriore fattore da tenere in considerazione riguarda le variabili atmosferiche: di natura non modificabili, richiedono un monitoraggio continuo e preciso. Viene definita \textit{evotraspirazione} il processo combinato attraverso il quale l'acqua viene trasferita nell'atmosfera, partendo da superfici d'acqua, dal ghiaccio, dal suolo e dalla vegetazione. Si tratta della combinazione tra \textit{evaporazione} e \textit{traspirazione}: la prima prevede il passaggio diretto dell'acqua nell'atmosfera e avviene per fattori come calore, umidità e velocità del vento; la seconda invece prevede il passaggio attraverso una pianta partendo dalle radici, fino a raggiungere le foglie, da dove avviene il passaggio all'atmosfera.
Le stazioni meteorologiche sono in grado di monitorare, in un dato momento, il \ac{$ET_0$} che rappresenta il valore di evotraspirazione in un dato momento di un ipotetica cultura di prato erboso in condizioni ottimali dove l'unico fattore limitante è dato dal clima. È possibile moltiplicare il dato \ac{$ET_0$} per un \ac{$K_c$} che dipende dal tipo e dallo stato di crescita di una determinata cultura, per ottenere \ac{$ET_a$}: reale evotraspirazione.
Questo permette quindi di approssimare la quantità di acqua usata da una pianta e che quindi deve essere somministrata tramite irrigazione, a patto che sia presente in letteratura una stima adeguata di $K_c$, che non è però sempre disponibile. Inoltre, \ac{$ET_0$} viene calcolato su un aria potenzialmente molto estesa e potrebbe quindi non essere adeguato per zone con zone ombreggiate o venti locali.
A ogni modo, il calcolo di \ac{$ET_a$} non tiene in considerazione le condizioni fisiche del suolo e dell'apparato radicale delle piante, per questo motivo il suo utilizzo esclusivo per decisioni riguardo l'irrigazione non è adeguato.

\subsection{Agricoltura di precisione}

Vista la mole di variabili, diventa interessante utilizzare sistemi di monitoraggio e analisi col fine di consigliare gli agricoltori a come correttamente gestire i campi. Per questo nasce l'agricoltura di precisione (\textit{precision agriculture}): consiste nell'uso di tecnologie avanzate per aumentare l'efficienza, la produzione e la qualità dei prodotti.
Primi esempi di agricoltura di precisione, nel modo in cui viene intesa a oggi, risalgono già agli anni '80, dove lo sviluppo di modelli di crescita delle culture e l'utilizzo di \ac{GPS} ha permesso di mappare i campi e monitorare variabili spaziali come fertilità del suolo ed esigenze idriche.
Negli anni '90, l'introduzione di sensori per il monitoraggio delle variabili ha permesso agli agricoltori di adattare le loro pratiche in base alle esigenze specifiche di ciascuna area del campo.
Oggi l'arrivo di nuove tecnologie come Cloud Computing, \ac{IoT} e intelligenza artificiale ha portato un aumento sostanziale delle ricerche nel campo.
Gli obbiettivi principali che si pone sono i seguenti:
\begin{itemize}
    \item \textbf{Aumentare l'efficienza produttiva}: usando le risorse in modo più mirato si è in grado di ridurre gli sprechi.
    \item \textbf{Migliorare qualità e resa delle culture}: grazie al monitoraggio continuo è possibile rilevare rapidamente problemi (come malattie, carenze nutrizionali o idriche) potendo così intervenire tempestivamente.
    \item \textbf{Ridurre l'impatto ambientale}: ottimizzando l'utilizzo di risorse è possibile diminuire l'uso di acqua e di fertilizzanti, riducendo quindi l'impatto ambientale.
    \item \textbf{Supportare la sostenibilità economica}: l'input minore richiesto dagli agricoltori, insieme con l'uso inferiore delle risorse, ha anche un effetto positivo dal punto di vista economico. Oltre a rendere la cultura più resiliente a variazioni climatiche.
\end{itemize}

\section{Componenti chiave dell’agricoltura di precisione}
\label{componenti-chiave}
Tecniche di misurazione e/o derivazione dei parametri
Modelli di simulazione del terreno
Sistemi di supporto alle decisioni - nel caso, qui puoi introdurre un gancio per parlare di AI in agricoltura (prossima sezione)

\section{Esempi di tecnologie e metodi avanzati nell’agricoltura di precisione}
\label{esempi-tecnologie}
Monitoraggio (!) - il nostro sistema prescrittivo è frutto di due anni di ricerca, ma già un sistema accurato di monitoraggio è un prodotto molto valido (e non così scontato) in industria agro
Machine learning e intelligenza artificiale per la gestione dell'irrigazione
Sistemi di irrigazione di precisione

\section{Limitazioni attuali dell’agricoltura di precisione}
\label{limiti}
Costi e accessibilità per i piccoli agricoltori
Sfide legate alla raccolta e gestione dei dati


\chapter{Tecnologie utilizzate}


%Descrive appunto le tecnologie utilizzate nel progetto: sia lato HW (Arduino + Raspberry + sensori), sia lato SW (stack tecnologico). Lunghezza 5-8 pagine.


\section{Hardware}

\begin{itemize}
    \item \textbf{Sensori}, TODO numero sensore e citazione alle specifiche
    \item \textbf{Arduino UNO R3} è una scheda microcontrollore basata sull'ATmega328P. Permette di caricare i programmi scritti in linguaggio C++ usando un'architettura SuperLoop, dove un unico pezzo di codice viene ripetuto fin tanto che il dispositivo è riceve alimentazione. Il superloop è interrompibile momentaneamente solo tramite degli interrupt di sistema, che vengono gestiti con routine non interrompibili, ciò implica che durante una routine, non possono avvenire altri interrupt. È quindi consigliabile che le routine di gestione siano particolarmente brevi, sia per evitare perdita di eventi, sia perchè molte librerie di sistema sfruttano internamente gli interrupt, per esempio \texttt{delay} (equivalmente a sleep di python). Offre l'utilizzo di un totale di 22 pin \ac{GPIO}: 16 dei quali digitali, di cui 6 utilizzabili come output analogici tramite \ac{PWM}, e altri 6 pin input analogico. Può inoltre gestire una comunicazione seriale tramite USB con la quale è possibile comunicare con altri dispositivi.
    \item \textbf{Raspberry Pi 4 model B} è un \ac{SBC} in grado di utilizzare sistemi operativi veri e propri basati sul kernel Linux o su RISC OS. Presenta un processore quad-core ARM a 64bit ad una velocità di clock di 1.80GHz. La configurazione utilizzata presenta \note{QUANTITARAM} Gb di RAM LPDDR4-3200. Compatibile con Bluetooth 5.0, BLE, connessioni wireless IEEE 802.11ac a 2.4 GHz e 5.0 GHz, oltre che Gigabit Ethernet. Permette anche l'utilizzo di 40 pin GPIO digitali. Compatibile con periferiche USB tramite 2 porte USB2.0 e 2 porte USB3.0. L'uscita video è permessa da tramite 2 porte Micro-HDMI, che supportano dual-screen 4k. Come storage viene utilizzata una singola scheda microSD. Include un'interfaccia MIPI DSI per display e MIPI CSI per telecamere, ed è alimentato tramite un connettore USB-C da 5V/3A. Prodotto dalla Raspberry Pi Foundation, che ha realizzato diversi modelli per varie esigenze, è compatibile con vari sistemi operativi, tra cui il più noto è Raspberry Pi OS (precedentemente chiamato Raspbian), una versione di Debian sviluppata dalla stessa fondazione.
\end{itemize}

\section{Software}

    \begin{itemize}
        \item \textbf{\ac{JS}} è un linguaggio di programmazione interpretato, usato principalmente per la programmazione web sia a livello client che server, per esempio con l'utilizzo di Node.js. È un linguaggio dinamico, basato sui prototipi, multi-paradigma, single-threaded , che supporta stili di programmazione orientati agli oggetti, imperativi e dichiarativi. Standardizzato come ECMAScript, \ac{JS} è alla base di un vasto ecosistema di librerie e framework che ne ampliano le capacità e l'applicabilità in diversi contesti.
        \begin{itemize}
            \item \textbf{Chart.js} è una libreria \ac{JS} semplice e flessibile per la generazione di grafici lato frontend. Si tratta di un progetto Open Source che sfrutta i canvas di HTML5. Di default permette l'utilizzo di 8 diversi tipi di grafici, è compatibile con plug-in che permettono l'aggiunta di nuove funzionalità.
            \begin{itemize}
                \item \textbf{chartjs-chart-matrix} è un plug-in che permette la creazione di grafici a matrice, un esempio è in \cref{fig.matrix_chart_example}
                \begin{figure}
                    \centering
                    \includegraphics[width=0.9\linewidth]{./figures/matrix-chart-example.png}
                    \caption{Esempio di una mappa delle temperature medie giornaliene per mese, l'intensità del colore è proporzionale alla temperatura.}
                    \label{fig.matrix_chart_example}
                \end{figure}
                \item \textbf{chartjs-plugin-streaming} è un plug-in che permette la creazione di grafici con valori in tempo reale. Prevede lo scorrimento verso sinistra del grafico in base al tempo. È possibile configurarlo per moficare la granularità, la frequenza di aggiornamento ed un eventuale delay in modo da avere una visualizzazione "fluida" dei nuovi dati. Necessita di un adapter per una libreria per la gestione delle date e ricezione del tempo attuale, come Luxon.
                \item \textbf{chartjs-plugin-datalabels} è un plug-in per la visualizzazione dei dati su un grafico, un esempio su un grafico a linea è visibile in \cref{fig.datalabels-example}
                \begin{figure}
                    \centering
                    \includegraphics[width=0.9\linewidth]{./figures/datalabels-example.png}
                    \caption{Esempio utilizzo di datalabels dove in ogni linea la posizione del label è differente.}
                    \label{fig.datalabels-example}
                \end{figure}
                \item \textbf{chartjs-adapter-luxon} è un plug-in che si occupa di fornire a chartjs-plugin-streaming le informazioni sulle date da Luxon.
            \end{itemize}
            \item \textbf{Luxon} è un wrapper di date e tempi per \ac{JS}. Utilizza un'API per gestiore le date, che consente il supporto agli intervalli, alle durate (14 days, 5 minutes, 33 seconds), la loror conversione e parsing. Offre la localizzazione delle stringhe con una gestione interna delle time-zones.
            \item \textbf{bootstrap} è un toolkit frontend potente ed estendibile. Permette la creazione di interfacce web utilizzando unicamente classi HTML, senza la necessità di scrivere il proprio foglio di stile CSS.
            \item \textbf{jQuery} è una famosa libreria per la manipolazione e navigazione di documenti HTML, la gestione degli eventi, le animazioni e richieste HTTP.
            \item \textbf{Socket.io} permette l'apertura di una comunicazione bidirezionale sempre attiva tra client e server utilizzando WebSocket.
        \end{itemize}
        \item \textbf{Python} è un linguaggio di programmazione ad alto livello che supporta il paradigma object-oriented, la programmazione struttrata e molte caratteristiche di programmazione funzionale. È un linguaggio debolmente tipizzato, che utilizza unicamente l'indentazione per la definizione dei blocchi di codice, al contrario di molti linguaggi, non prevede quindi l'utilizzo di parentesti graffe. Supporta l'overloading di operatori e funzioni tramite delegati, oltre che sinstassi avanzate quali slicing (l'utilizzo di un certo range di elementi in una lista) e list comprehension (la creazione di nuove liste partendo da una originale). Il codice python viene interpretato (con delle compilazioni a bytecode alla prima esecuzione per migliorare le performance) che supporta anche un uso interattivo.
        \begin{itemize}
            \item \textbf{Flask} è un framework web per la creazione di backend per Python.
            \item \textbf{Flask-SocketIO} è una libreria Python per la gestione di websocket tramite Socket.io.
            \item \textbf{NumPy} è la libreria principale per il calcolo scientifico Python. Permette la gestione di array $n$-dimensionali ed è ampiamente utilizzato per operazioni di algebra lineare, trasformate di Fourier, manipolazioni di dati e altre applicazioni numeriche. È uno strumento essenziale per chi lavora con analisi numeriche, machine learning e data science.
            \item \textbf{pySerial} è una libreria per l'accesso a porte seriali. Permette la stessa gestione delle porte seriali su ogni piattaforma. Supporta diversi byte size, bit di parità e di fine messaggio, oltre che controllo del flusso. La porta è configurata per la trasmissione binaria.
            \item \textbf{dotenv} è un modulo per la gestione delle variabili d'ambiente.
            \item \textbf{pytz} è una libreria per il calcolo del fuso orario in python.
            \item \textbf{SciPy} è una libreria di Python per il calcolo scientifico e tecnicho, espande NumPy. Offre un'ampia quantità di funzionalità per l'ottimizzazione, la statistica, l'integrazione, l'algebra lineare, l'elaborazone dei segnali, la gestione di immagini e altre operazioni matematiche avanzate. È molto utilizzata per esempio nei campi della fisica, dell'ingegneria, della biologia computazionale.
        \end{itemize}
        \item \textbf{C++} è un linguaggio di programmazione nato come espansione del linguaggio C, presenta caratteristiche di programmazione funzionale, generica e orientata agli oggetti, permettendo però una gestione a basso livello della memoria. Per quest'ultima caratteristica è tipicamente utilizzato in contesti dove le prestazioni sono particolamente importanti come lo sviluppo di sistemi operativi, software embedded, motori di gioco e applicazioni real-time. In particolar modo è utilizzanto anche per lo sviluppo di software per dispositivi con risorse limitate, come microcontrollori e sistemi embedded, dove la gestione diretta della memoria e delle risorse hardware è essenziale.
        \begin{itemize}
            \item \textbf{Arduino API} è una libreria in C/C++ per la scrittura di codice per dispositivi Arduino. Contiene funzioni riguardanti l'hardware, per esempio, contiene funzioni per la lettura/scrittura di segnali digitali e analogici sui pin, per la gestione degli interrupt e della porta seriale. Oltre alle funzioni hardware, l'API include, ad esempio, strumenti per la gestione dei timer, funzioni matematiche, funzioni per la gestione di bit e byte e per la gestione degli stream. Contiene inoltre i tipi di variabile `enum`, `String` e funzioni per la conversione.
            \item \textbf{TimerInterrupt} è una libreria che permette l'utilizzo del timer fisico di una board Arduino come interrupt di sistema, permettendo lo svolgimento ininterrotto di ruotine periodicamente.
            \item \textbf{ArduinoJson} è una semplice libreria per la gestione di stringhe JSON su piattaforma Arduino. Permette serializzazione e deserializzazione, ed è quindi molto utile per mandare dati strutturati sul seriale in modo che venga che lo scambio dei dati sia standardizzato.
        \end{itemize}
    \end{itemize}



\chapter{Un prototipo di irrigazione prescrittivo}
Descrive nel dettaglio il tuo lavoro, quindi come le tecnologie presentate nella precedente sezione sono state utilizzate nel progetto. Puoi prenderla larga descrivendo il nostro sistema (quello che ti ho mostrato sul mio pc) per poi descrivere la necessità di averne una rappresentazione su piccola scala e come è stata realizzata, così come descrivere il ciclo di vita del dato (collection, processing, exploitation). Ora come ora è normale tu non abbia chiare le idee su questo capitolo, andando avanti integreremo le tue conoscenze con quelle pregresse nostre sul dominio applicativo e sul nostro sistema. Questo capitolo deve essere il core della tesi, lunghezza 20 pagine.

\chapter{Conclusioni e sviluppi futuri}

Breve capitolo che trae le conclusioni sul lavoro svolto, il risultato ottenuto rispetto a quello atteso e lo spazio dedicato a migliorie future.

%
% BIBLIOGRAPHY
%

\backmatter

\bibliographystyle{alpha}
\bibliography{bibliography}

\begin{acknowledgements} % this is optional
    Optional. Max 1 page.
\end{acknowledgements}

\end{document}
